% Includes definitions for single_histogram.tex.erb and only needs
% to be included once. 

% Defines the histogram for single_histogram.tex.erb. Assumes “leftmost
% pole” is already present in the TikZ picture. The values should be in
% pgfplot compatible coordinates format. x-value defines the position,
% y-value the percentage of that answer. Usage:
% \singleHistogramHistStyle{#answers}{values; style: (1,15) (2,50), …}
\newcommand{\singleHistogramHist}[1]{
  % add a little space to the right of the leftmoste pole. For some
  % reason the right= syntax does not work properly on the pfgplots.
  \path (leftmost pole.east) ++(0.2cm,0) coordinate (leftmost pole end);

  \StrCount{#1,}{,}[\argumentCount]
  % pgfplots doesn’t support expressions in xmax, therefore calculate
  % the width of the diagram here.
  \FPadd{\calcxmax}{\argumentCount}{0.5}

  % generate the coordinates which will later be parsed. The x-value
  % is simply counted up while the y-value is taken from #1.
  \gdef\coordString{}
  \foreach \x in {1,...,\argumentCount}
    \findNthArgument{#1}{\numexpr\x-1}
    \xdef\coordString{\coordString (\x, \lastFindNthArgument) };

  \begin{axis}[
    name=histogram,
    % place left of leftmost node
    at={(leftmost pole end)},
    anchor=west,
    % set style to bar plot, set ranges and width
    ybar, ymin=0, ymax=100,
    xmin=0.5, xmax={\calcxmax},
    width=7cm, height=2.5cm,
    % 1.6cm are used for… space, divide the rest between each plot
    bar width = (7cm-1.6cm)/\argumentCount,
    % show tick (and label) for each data point
    xtick=data,
    % disable y-ticks and labels, the axis should be clear
    ytick=\empty,
    % don’t know how to write y-data as label automatically :(
    xticklabel={\findNthArgument{#1}{\ticknum}\%},
    % put these labels above the plot and move them closer
    xticklabel pos=right,
    every x tick label/.style={yshift=-0.2cm},
    % make them a little bit smaller
    tick label style={font=\footnotesize},
    % no magic with the borders, use x/y-min/max values
    enlargelimits=false,
    % this draws the lines between the major ticks
    xminorgrids, minor x tick num=1,
    every minor grid/.append style={color=black},
    % disable little grey markers on top or below the actual plot
    major x tick style={draw=none},
    minor x tick style={draw=none},
  ]
  % addplot is rather picky about its coordinate input. Therefore expand
  % them without addplot knowing and then execute.
  \edef\temp{\noexpand\addplot[black,fill=black] coordinates {\coordString};}
  \temp
  \end{axis}
}

% Defines the error bars for single_histogram.tex.erb. Assumes the
% histogram is already present in the TikZ picture. Usage:
% \singleHistogramErr{data avg}{data stddev}{comparison avg}{cmp stddev}
\newcommand{\singleHistogramErr}[4]{
  % Define position where to place the avg/stddev plot. It’s currently
  % below the histogram, but moved up slightly.
  \path (histogram.below south west) ++(0,0.1cm)
          coordinate (just below histogram);

  \begin{axis}[
    % make this a scatter plot, but only two points will be used. “this”
    % describes the style for avg/stddev for this data, “comp” for the
    % comparison value used.
    scatter/classes={
      this={mark=*,draw=black},
      comp={mark=*,draw=black,fill=white}},
    % position it correctly
    at={(just below histogram)},
    anchor=north west,
    % same width as for the histogram.
    width=7cm, height=2.1cm,
    % match the width of histogram. I.e. 0.5 to (#of answers)+0.5. We
    % assumed the histogram has been printed, therefore \calcxmax should
    % contain the correct value.
    xmin=0.5, xmax=\calcxmax,
    % disable axis ticks and labels
    xtick=\empty, ytick=\empty,
    axis x line = none, axis y line = none,
    % chosen so that the vertical bars at the end of each error bar fit
    % in the diagram.
    ymin=0, ymax=3]
    % make it a scatter plot, because the coordinates are not related.
    % x value is the average, y value is fixed and only provides order
    % and space between bars.
    \addplot[scatter, only marks]
      % style error bars using classes above
      plot[scatter src=explicit symbolic,error bars/.cd,x dir=both,
            x explicit]
      coordinates {
        %AVG POS  STDDEV   STYLE
        (#1, 2) +- (#2, 0) [this]
        (#3, 1) +- (#4, 0) [comp]
      };
  \end{axis}
}

% Defines the left pole for single_histogram.tex.erb. Assumes the
% question text is already present in the TikZ picture. Usage:
% \singleHistogramPoleLeft{pole text}
\newcommand{\singleHistogramPoleLeft}[1]{
  \draw node[right=2mm of question text.east, fill=red, text width=2cm,
              text ragged left, inner sep=0] (leftmost pole)
    {\footnotesize #1};
}

% Defines the right pole for single_histogram.tex.erb. Assumes the
% histogram is already present in the TikZ picture. Usage:
% \singleHistogramPoleRight{pole text}
\newcommand{\singleHistogramPoleRight}[1]{
  \draw node[right={2mm of histogram.east}, fill=red, text width=2cm,
              text ragged, inner sep=0] (rightmost pole)
    {\footnotesize #1};
}
