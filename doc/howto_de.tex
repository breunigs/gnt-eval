\documentclass[a4paper,10pt]{article}
\usepackage[margin=3cm]{geometry}
\usepackage{ngerman}
\usepackage{helvet}
\usepackage[utf8]{inputenc}
\pagestyle{empty}
\begin{document}
\subsection*{Howto Eval}
Dir ist freiwillig oder durch gewissen Druck die Ehre zu Teil
geworden, eine Vorlesung zu evaluieren. Eigentlich ist das
kinderleicht und macht eventuell sogar Spaß. Wenn du die folgenden
Punkte beachtest, kann nicht viel schief gehen und die Evaldeppen
werden dir dankbar sein, da sie so weniger Arbeit haben.

\subsection*{12 Schritte zu einer erfolgreichen Eval}
\begin{enumerate}
\item Lies die Anleitung \textbf{vorher} und \textbf{ganz zu Ende}
\item Geh \textbf{rechtzeitig} vor der Vorlesung in den FS Raum und hol die
  Mappe ab auf der der Name deiner Vorlesung steht!
\item Geh vor der Vorlesung zum/r DozentIn und stell dich als
  Fachschaftler vor.
\item Der Dozent wurde über dein Kommen informiert -- dennoch höflich
  bleiben, wenn er nichts von der Eval weiß.
\item Führe die Eval am \textbf{Vorlesungsanfang} durch (oder was Du mit dem Prof ausgemacht hast)
\item Halte eine kurze Ansprache -- evtl. per Mikrophon -- an die
  Hörer, in der du folgende Dinge erwähnst:
    \begin{itemize}
    \item Warum Eval? Feedback für Prof, Verbesserung der Lehre
    \item Wer macht die Eval? FS in Zusammenarbeit mit der
          Studienkommission.
    \item Wo findet man die Eval-Ergebnisse? FS Raum; in den
          Instituten
    \item \textbf{deutliche Kreuze} machen (Kugelschreiber, Füller, keine
          hellen Fineliner)
    \item auf die automatische Auswertung hinweisen! -- Hinweise
          auf dem Bogen sind nutzlos, Kreuze, wo keine Felder sind,
          ebenfalls
    \item Textfelder \textbf{in Stichworten} und \textbf{ordentlich} ausfüllen. Außerdem: Aussagen sollten in sich stimmig sein, z.B. wäre „Skript“ kein guter Kommentar, „gutes Skript“ oder „Skript wäre gut“ hingegen schon.
    \end{itemize}
\item Bögen verteilen und auf Rückfragen eingehen
\item Ausreichend Zeit zum Ausfüllen lassen! Wenn du merkst, dass der
  Dozent unruhig wird, frag ruhig die Hörer, ob sie noch Zeit
  benötigen. Während der Eval findet keine Vorlesung statt!
\item Bögen vollständig einsammeln! \textbf{Kein Nachreichen} erlauben -- das
  ist extrem wichtig!
\item beim Dozenten bedanken und verabschieden
\item Zurück im FS Raum leere Bögen \textbf{aussortieren und vernichten}: mit
  der Schneidemaschine halbieren (!)
\item ausgefüllte Evalbögen in den Karton „ausgefüllte Bögen“, den Du durch Suchen findest.
\item Fertig!
\end{enumerate}

\subsection*{Lustige Probleme}
\noindent\textbf{zu wenig Bögen:} GAU! Geh sofort in den FS Raum oder ruf dort an (06221/54-4167) und
druck Bögen oder lass sie drucken! Diese liegen unter: \hfill  Auf keinen Fall andere Bögen verwenden!\\
\textit{§§§/„Vorlesungsname -- Profname“.pdf}\\[0.2cm]
\noindent\textbf{mehrere Dozierende:} Teile die Bögen \emph{nacheinander} und schreibe an, wer gerade geevalt wird.\\[0.2cm]
\noindent\textbf{ein Tutorium, mehrere Tutorierende:} Teile einen Bogen pro Tutor aus. Der Vorlesungsteil sollte nur einmal ausgefüllt werden.

\end{document}
